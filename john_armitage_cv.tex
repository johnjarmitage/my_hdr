%%%%%%%%%%%%%%%%%%%%%%%%%%%%%%%%%%%%%%%%%
% Wilson Resume/CV
% XeLaTeX Template
% Version 1.0 (22/1/2015)
%
% This template has been downloaded from:
% http://www.LaTeXTemplates.com
%
% Original author:
% Howard Wilson (https://github.com/watsonbox/cv_template_2004) with
% extensive modifications by Vel (vel@latextemplates.com)
%
% License:
% CC BY-NC-SA 3.0 (http://creativecommons.org/licenses/by-nc-sa/3.0/)
%
%%%%%%%%%%%%%%%%%%%%%%%%%%%%%%%%%%%%%%%%%

\chapter{Rèsumé}

%----------------------------------------------------------------------------------------
%	EDUCATION SECTION
%----------------------------------------------------------------------------------------

\section{Education}

\tabbedblock{
\bf{2004-2009} \> PhD in Geophysics - National Oceanography Centre, University of Southampton, UK \\[5pt]
\>Title `How Does Melting Affect Continental Rift Development?' \\
\>Supervisors: Prof Tim Minshull, Dr Tim Henstock \& Dr John Hopper\\
\>\+
\textit{Awarded a fully funded NERC studentship}
}

%------------------------------------------------

\tabbedblock{
\bf{2003-2004} \> Masters in Oceanography - National Oceanography Centre, University of Southampton, UK \\[5pt]
\>Masters Project Title `Dissolution of Silica from Antarctic Continental Shelf Sea Sediments'\\ \>Supervisor: Prof. Rachel Mills\\
\>\+
\textit{Awarded a fully funded NERC scholarship}
}

%------------------------------------------------

\tabbedblock{
\bf{1999-2003} \> MSci in Physics - Imperial College London, UK \\%[5pt]
%\>Upper Second Class Honours- 79\% Average\\
\>\+
}

%----------------------------------------------------------------------------------------
%	EMPLOYMENT HISTORY SECTION
%----------------------------------------------------------------------------------------

\section{Employment History}

\job
{Feb 2020 -}{Present}
{IFP Energies Nouvelles}
{http://www.ifpenergiesnouvelles.fr}
{Ingènieur de Recherche}
{}

%------------------------------------------------

\job
{Oct 2019 -}{Feb 2020}
{Gekko SAS, Paris, France}
{http://www.gekko.fr}
{Cloud Computing and DevOps Consultant}
{}

%------------------------------------------------

\job
{Sep 2015 -}{Aug 2019}
{Institut de Physique du Globe de Paris, France}
{http://www.ipgp.fr}
{ANR Funded Research Scientist}
{I was the principle investigator on a prestigious 4 year Agence National de la Recherche funded grant through the “Acceuil de Chercheurs de Haut Niveau” call.}

%------------------------------------------------

\job
{Sep 2013 -}{Aug 2015}
{Department of Earth Sciences, Royal Holloway, University of London, UK}
{https://www.royalholloway.ac.uk/earthsciences/home.aspx}
{Royal Astronomical Society Research Fellow}
{This fellowship, titled “Deciphering the sedimentary record: tectonic vs climate change”, was awarded by the Royal Astronomical Society. I explored how long-wavelength topography changes due to mantle flow, and whether long lived changes in climate have affected the sedimentary record of continental interiors I am continuing collaborations with Prof.~Jason Phipps Morgan, Prof.~Peter Burgess and Prof.~Marta P{\'e}rez-Gussiny{\'e} to couple numerical models of mantle flow to first order sedimentary processes.}

%------------------------------------------------

\job
{Apr 2011 -}{Aug 2013}
{Institut de Physique du Globe de Paris, France}
{http://www.ipgp.fr}
{Marie Curie Research Fellow}
{This two year EU Marie Curie fellowship was titled “Cratonic basins: an archive of lithosphere-mantle interaction”. Collaborating with Prof.~Claude Jaupart (IPGP); I studied how deep mantle dynamics is recorded in the rock record by exploring the relationship between lithosphere destabilisation and subsidence within the continental interior, with a focus on cratonic basins. The primary aim was to develop my numerical and laboratory experimental skills.}

%------------------------------------------------

\job
{Mar 2008 -}{Mar 2011}
{Department of Earth Science and Engineering, Imperial College London, UK}
{https://www.imperial.ac.uk/earth-science/}
{Research Associate}
{I worked within the Sediment Routing Systems group, headed by Prof. Philip Allen, to (1) Develop models of basin formation and couple models of subsidence with the evolution of grain size. In particular I looked at the possibility of subsidence within the continental interior at cratonic basins being due to a low strain rate extension. (2) Develop models of catchment erosion and sediment deposition to explore the controls of basin stratigraphy due to changes in climate and uplift within the catchment.}

%----------------------------------------------------------------------------------------
%	FUNDING
%----------------------------------------------------------------------------------------

\section{Funding}

\tabbedblock{
\bf{Sep 2015} \> ANR Acceuil de Chercheurs de Haut Niveau 4 year research project \\
\> `InterRift: Interpreting Continental Break-up from Surface Observations' -- 450,000\euro \\
\bf{Aug 2013} \> Royal Astronomical Society Research Fellowship -- \pounds123,000 \\
\bf{Apr 2011} \> E.U.~Marie Curie Research Fellowship -- 193,564\euro \\
}

%----------------------------------------------------------------------------------------
%	AWARDS
%----------------------------------------------------------------------------------------

\section{Awards}

\tabbedblock{
\bf{2011} \> Young Author of the Year, Journal of the Geological Society, London, for the article\\
\> “Cratonic basins and the long-term subsidence history of continental interiors”.\\
\bf{2009} \> Masanori Sakuyama Prize, for best publication of a PhD graduate of the National \\
\> Oceanography Centre, Southampton: `Modelling the composition of melts formed during the\\
\> continental break-up of the North Atlantic', published in Earth and Planetary Science Letters.\\
\bf{2007} \> Best poster at the Faculty of Engineering, Science \& Mathematics Research\\
\> Showcase, University of Southampton.\\
\bf{2003} \> NERC scholarships including living costs for my Masters in Oceanography \& my PhD.\\
}

%----------------------------------------------------------------------------------------
%	PhD SUPERVISION
%----------------------------------------------------------------------------------------

\section{PhD Supervision}

\job
{2016 -}{2019}
{Thijs Franken}
{}
{Institut de Physique du Globe de Paris}
{I was the primary supervisor of Thijs Franken. Thijs' PhD focused on developing models of seismic structure due to partial melt and wave propagation through regional domains.  This project was funded through my ANR grant, and was in collaboration with Dr.~Nobuaki Fuji and Prof.~Alexandre Fournier (IPGP).}

\job
{2014 -}{2018}
{Sam Brooke}
{}
{Imperial College London}
{I co-supervised Sam Brooke with Dr.~Alex Whittaker at Imperial College London. The aim of this project was to further develop models of sediment transport and the evolution of alluvial fans.}

\job
{2012 -}{2017}
{Chandra Taposeea}
{}
{Imperial College London}
{I co-supervised Chandra Taposeea with Dr.~Jenny Collier at Imperial College London. The project further tests the hypothesis developed by myself, Jenny Collier and Tim Minshull (National Oceanography Centre, Southampton) that rift history controls the volume of melt generated during continental break-up.}

%----------------------------------------------------------------------------------------
%	MASTERS SUPERVISION
%----------------------------------------------------------------------------------------
\section{Masters and Undergraduate Supervision}

\tabbedblock{
\bf{2017} \> Masters research project: Aimen Maghrebi, \textit{Ecole Supérieure d’Ingénieurs Léonard de Vinci}.\\
\bf{2017} \> 3rd year undergraduate project: Sabrina Ihaddadene, License, \textit{Physique, Université Paris-Est Créteil}.\\
\bf{2016} \> 3rd year undergraduate project: Leo Bourier, License, \textit{Physique, Université Paris-Est Créteil}.\\
}

%----------------------------------------------------------------------------------------
%	TEACHING
%----------------------------------------------------------------------------------------
\section{Teaching}

\tabbedblock{
\bf{2018} \> Université Paris Diderot, Surface Processes, Masters level, 50h\\
\bf{2017} \> Université Paris Diderot, UE Projet Tutoré, 3rd year undergraduate, 42h\\
\bf{2016} \> Institut de Physique du Globe de Paris, `Frontiers in Earth surface dynamics: creating a\\
\> research proposal', École doctorale (PhD level), 10h\\
\bf{2016} \> Université Pierre et Marie Curie, Séminaire Géodynamique, Masters level, 3h\\
\bf{2014} \> Royal Holloway, University of London, Sedimentary Basin Analysis, 3rd year undergraduate, 6h\\
\bf{2010} \> Imperial College London, Mathematics of Geology and Geophysics, 1st year undergraduate, 20h\\
\bf{2007-2008} \> University of Southampton, Biogeochemical cycles, Masters level, 20h\\
}

%----------------------------------------------------------------------------------------
%	SELECTED CONFERENCE PRESENTATIONS
%----------------------------------------------------------------------------------------

\section{Selected Invited Presentations}

{\bf EGU General Assembly 2018}: From induction to deduction: Using the Earth as a natural laboratory. EGU2018-10469, April, 2018.\\
{\bf Ordinary Meeting of the RAS}: Can variations in the Earth's orbit create stratigraphic sequences? 10th March 2017\\
{\bf Geological Society of London}, Rifts III: Catching the wave, 2016: Upper mantle temperature during extension and breakup, 22-24 March, 2016.\\
{\bf AGU Fall Meeting 2015}: Testing How Depletion, Dehydration and Melt Affect Seismic Expressions of the Asthenosphere, T34C-01, December 16th, 2015.\\
{\bf Volkswagen-Stiftung Funded Symposium}: Bridging the Gap Between Field Evidence and Numerical Models: Modelling landscape and sediment flux responses to precipitation change, Herrenhausen Palace Conference Centre, Hanover, October 21-23rd, 2015.\\
{\bf Action Marges Workshop, Mouvements verticaux et Chantier Afar-Aden}: Keynote: Controls of lithospheric extension on magma and sediment production, Total, La Defence, November 11-12th, 2014.\\
{\bf Volcanic \& Magmatic Studies Group 2012 Conference}: Keynote: Beyond Hotspots: the importance of rift history for volcanic margin formation, Durham University, UK, January 4-6th, 2012.\\
{\bf EarthScope - GeoPRISMS Science Workshop on Eastern North America (ENAM)}: Analogue and numerical models that inform the rifting process, Lehigh University, Bethlehem, PA, USA, October 27-29th, 2011.\\

\section{Selected Oral Conference Presentations}
Armitage, J.J. (2017), The Effect of Climate Change on Volcanism During Continental Break-up, 79th EAGE Conference and Exhibition 2017.\\
Armitage, J.J. (2016), But What About Trees and Beavers? Simplicity, Complexity and Benchmarks for Landscape Evolution Models,  AGU, Fall Meet., Abstract, EP12A-07\\
Armitage, J.J. (2015), Landscape response due to sediment transport and bedrock detachment, GeoBerlin, Dynamic earth – from Alfred Wegener to today and beyond, Berlin.\\
Armitage, J.J. (2014), The influence of large-scale tectonics, mantle flow and climatic change on sediment accumulation, William Smith Meeting 2014: The Future of Sequence Stratigraphy: Evolution or Revolution? The Geological Society, London.\\
Armitage J.J., Barrier, L. (2013), Is the Neogene series of the Northern Foreland Basin of the Tian Shan Range indicative of tectonic or climatic change? AGU, Fall Meet., Abstract EP43E-04 \\
Armitage J.J. (2012), The Instability of Continental Passive Margins and its Effect on Continental Topography and Heat Flow, Deep Water Margins, The Geological Society, London.\\
Goes, S.D., Armitage, J.J., Harmon, N., Huismans, R.S., (2011), Low seismic velocities below mid-ocean ridges: Attenuation vs. melt retention, AGU, Fall Meet., Abstract, T33H-05 \\
Armitage J.J., Duller, R.A., Whittaker, A.C., Densmore, A, Allen, P.A., (2010) Response of sediment routing systems to tectonic and climatic perturbations, William Smith Meeting 2010 - Landscapes Into Rock, The Geological Society, London.\\
Armitage J.J., Collier, J.S., Minshull T.A., Hopper J.R., (2008), Geodynamic modelling of the opening of the northwest Indian Ocean, AGU, Fall Meet., Abstract, T53G-08\\
Armitage J.J., Henstock T.J., Minshull T.A., Hopper J.R., (2007), Lithospheric controls on the rifting of continents at slow rates of extension, AGU, Fall Meet. Suppl., Abstract, T32A-06\\

\section{Invited Seminars}
More than 20 invited seminars since 2012.
\tabbedblock{
\bf{2018} \> University of Leeds \\
\bf{2017} \> ISTerre Grenoble; GET Toulouse; IRAP Toulouse \\
\bf{2016} \> Universié de Lausanne; GFZ Potsdam \\ 
\bf{2015} \> University of Edinburgh; MINES ParisTech; University of Southampton \\
\bf{2014} \> University of Southampton \\
\bf{2013} \> Université de Montpelier; Université de Nantes; Université Paris Sud, Orsay \\
\bf{2012} \> ENS, Paris; FAST, Université Paris Sud, Orsay; CPRG Nancy; UMPC, Paris; Université de Rennes;\\
          \> EOST Strasbourg \\
}

%----------------------------------------------------------------------------------------
%	PUBLICATIONS
%----------------------------------------------------------------------------------------

\section{List of publications}

\pub {36}{ Franken, T, {\bf Armitage, J.J.}, Fuji, N, Fournier, A., (2020) Seismic wave-based constraints on geodynamical processes: an application to partial melting beneath the Réunion island. Geochemistry, Geophysics, Geosystems, doi: TBA}

\pub {35}{ Civiero, C., {\bf Armitage, J.J.}, Goes, S., Hammond, J.O.S., (2019) The seismic signature of upper‐mantle plumes: application to the northern East African Rift. Geochemistry, Geophysics, Geosystems, 20, 6106-6122, doi: 10.1029/2019GC008636}
\pub {34}{ Andrés-Mart\'inez, M., Pérez-Gussinyé, M., {\bf Armitage, J.J.}, Morgan, J.P., (2019) Thermomechanical implications of sediment transport for the architecture and evolution of continental rifts and margins. Tectonics, 38, 641-665, doi: 10.1029/2018TC005346}
\pub {33}{ Duller, R.A., {\bf Armitage, J.J.}, Manners, H.R., Grimes, S., Jones, T.D., (2019) Delayed sedimentary response to abrupt climate change at the Paleocene-Eocene boundary, northern Spain. Geology, 47, 159-162, doi: /10.1130/G45631.1}
\pub {32}{ {\bf Armitage, J.J.} (2019) Short communication: flow as distributed lines within the landscape. Earth Surface Dynamics, 7, 67-75, doi: 10.5194/esurf-7-67-2019}
\pub {31}{ {\bf Armitage, J.J.}, Ferguson, D.J., Petersen, K.D., Creyts, T.T., (2019) The importance of Icenlandic ice sheet growth and retreat on mantle CO$_2$ flux. Geophyscial Research Letters, 46, 6451-658, doi: 10.1029/2019GL081955}
\pub {30}{ {\bf Armitage, J.J.}, Collier, J.S., (2017) The thermal structure of volcanic passive margins. Petroleum Geoscience, 24, 393-401, doi: 10.1144/petgeo2016-101}
\pub {29}{ Brooke, S.A.S., Whittaker, A.C., {\bf Armitage, J.J.}, D'Arcy, M., Watkins, S.E., (2018) Quantifying sediment transport dynamics on alluvial fans from spatial and temporal changes in grain size, Death Valley, California. Journal of Geophysical Research -- Earth Surface, 123, 2039-2067, doi: 10.1029/2018JF004622}
\pub {28}{ Rychert, C.A., Harmon, N., {\bf Armitage, J.J.}, (2018) Seismic imaging of thickened lithosphere resulting from plume pulsing beneath Iceland. 19, 1789-1799, doi: 0.1029/2018GC007501}
\pub {27}{ {\bf Armitage, J.J.}, Petersen, K.D., Perez-Gussinye, M., (2018) The role of crustal strength in controlling magmatism and melt chemistry during rifting and break-up. Geochemistry Geophysics Geosystems, doi: 10.1002/2017GC007326}
\pub {26}{ {\bf Armitage, J.J.}, Burgess, P.M.,  Hampson, G.J., Allen, P.A., (2018) Deciphering the origin of cyclical gravel front and shoreline progradation and retrogradation in the stratigraphic record. Basin Research, 30, 15-35, doi: 10.1111/bre.12203}
\pub {25}{ {\bf Armitage, J.J.}, Whittaker, A.C., Zakari, M., Campforts, B., (2018) Numerical modelling landscape and sediment flux response to precipitation rate change. Earth Surface Dynamics, 6, 77-99, doi: 10.5194/esurf-6-77-2018}
\pub {24}{ Taposeea, C.A., {\bf Armitage, J.J.}, Collier, J.S., (2017) Asthenosphere and lithosphere structure controls on early onset oceanic crust production in the southern South Atlantic. Tectonophysics, 716, 4-20, doi 10.1016/j.tecto.2016.06.026}
\pub {23}{ Mareschal, J.-C., Jaupart, C., {\bf Armitage, J.J.}, Phaneuf, C., Pickler, C.,  Bouquerel, H., (2017) The Sudbury Huronian Heat Flow Anomaly, Ontario, Canada. Precambrian Research, 295, 187-202, doi: 0.1016/j.precamres.2017.04.024}
\pub {22}{ Allen, P.A., Michael, N.A., D’Arcy M., Roda-Boluda, D.C., Whittaker, A.C., Duller, R.A, {\bf Armitage, J.J.}, (2017) Fractionation of grain size in terrestrial sediment routing systems. Basin Research, 29, 180-202, doi: 10.1111/bre.12172}
\pub {21}{ Temme, A.J.A.M., {\bf Armitage, J.J.}, Attal, M., van Gorp, W., Coulthard, T.J., Schoorl, J.M., (2017) Developing, choosing and using landscape evolution models to inform field-based landscape reconstruction studies. Earth Surface Processes and Landforms, 42, 2167–2183, doi: 10.1002/esp.416}
\pub {20}{ Duller, R.A., Warner, N.H., De Angelis, S., {\bf Armitage, J.J.}, Poyatos-More, M., (2015) Reconstructing the timescale of a catastrophic fan-forming event on Earth using a Mars model. Geophysical Research Letters, 42, 10324-10332, doi: 10.1002/2015GL066031}
\pub {19}{ {\bf Armitage, J.J.}., Allen, P. A., Burgess, P. M., Hampson, G. J., Whittaker, A. C., Duller, R. A., Michael, N. A., (2015) Physical stratigraphic model for the Eocene Escanilla sediment routing system: Implications for the uniqueness of sequence stratigraphic architectures. Journal of Sedimentary Research, 85, 1510-1524, doi:10.2110/jsr.2015.97}
\pub {18}{ Allen, P.A., {\bf Armitage, J.J.}, Whittaker, A.C., Michael, N.A., Roda-Boluda, D., D’Arcy, M., (2015) Fragmentation model of the grain size mix of sediment supplied to basins. Journal of Geology, 123, 405-427, doi: 10.1086/683113}
\pub {17}{ {\bf Armitage, J.J.}, Ferguson D., Goes, S., Hammond, J.O.S., Calais, E., Harmon, N., Rychert, C.A., (2015) Upper mantle temperature and the onset of extension and break-up in Afar, Africa. Earth and Planetary Science Letters, 418, 78-90, doi: 10.1016/j.epsl.2015.02.039}
\pub {16}{ Lucazeau, F., {\bf Armitage, J.J.}, Kadima Kabongo, E., (2015) Thermal regime and evolution of the intracratonic Congo Basin, in The Geology and Resource Potential of the Congo Basin , de Wit, M.J., Guillocheau, F., Fernadez-Alonso, M., Kanda, N., and de Wit, M.C.J., Eds. Elsevier. doi: 10.1007/978-3-642-29482-2\_12}
\pub {15}{ Petersen, K.D., {\bf Armitage, J.J.}, Nielsen, S.B., Thybo, H., (2015) Mantle temperature as primary control on the time scale of thermal evolution of extensional basins, Earth and Planetary Science Letters, 409, 61-70, doi: 10.1016/j.epsl.2014.10.043}
\pub {14}{ {\bf Armitage, J.J.}, Duller, R.A., Schmalholz, S.M., (2014) The influence of long-wavelength tilting and climatic change on sediment accumulation. Lithosphere, 6, 303-318, doi: 10.1130/L343.1}
\pub {13}{ {\bf Armitage, J.J.}, Dunkley Jones, T., Duller, R.A., Whittaker, A.C., Allen, P.A., (2013) Temporal buffering of climate-driven sediment flux cycles by transient catchment response. Earth and Planetary Science Letters, 369, 200-210, doi: 10.1016/j.epsl.2013.03.020}
\pub {12}{ {\bf Armitage, J.J.}, Jaupart, C., Fourel, L., Allen, P.A. (2013) The instability of continental passive margins and its effect on continental topography and heat flow. Journal of Geophysical Research – Solid Earth, 118, 1817–1836, doi: 10.1002/jgrb.50097}
\pub {11}{ Allen, P.A., {\bf Armitage, J.J.}, Carter, A., Duller, R.A., Michael, N., Sinclair, H.D., Whitchurch, A.L., Whittaker, A.C. (2013) The Qs problem: Sediment mass balance of proximal foreland basin systems. Sedimentology, 60, 102-130, doi: 10.1111/sed.12015}
\pub {10}{ Goes, S., {\bf Armitage, J.J.}, Harmon, N., Smith, H., Huismans, R., (2012) Low seismic velocities below mid-ocean ridges: Attenuation vs. melt retention. Journal of Geophysical Research – Solid Earth, 117(B12403), doi: 10.1029/2012JB009637}
\pub {9}{ Duller, R.A., Whittaker, A.C., Swinehart, J.B., {\bf Armitage, J.J.}, Sinclair, H.D., Bair, A.R., Allen, P.A., (2012) Abrupt landscape change post-6 Ma on the Central Great Plains, U.S.A. Geology, 40, 871-874}
\pub {8}{ Allen, P.A. \& {\bf Armitage, J.J.}, (2012) Cratonic basins. In Busby, C. \& Azor, A. (Eds.) Syntectonic Basin Development, Active to Ancient: Recent Advances. Ch. 30, p 602-620, Blackwell Publishing Ltd.}
\pub {7}{ {\bf Armitage, J.J.}, Warner, N.H., Goddard, K., Gupta, S., (2011) Timescales of alluvial fan development on Mars. Geophysical Research Letters , 38 (L17203), doi:10.1029/2011GL048907}
\pub {6}{ {\bf Armitage, J.J.}, Collier, J.S., Minshull, T.A. Henstock, T.J. (2011) Thin oceanic crust and flood basalts: Reconciling observations from the northwest Indian Ocean. Geochemistry Geophysics Geosystems, 12(Q0AB07), doi:10.1029/2010GC003316}
\pub {5}{ {\bf Armitage, J.J.}, Duller, R.A., Whittaker, A.C. Allen, P.A., (2011) Transformation of tectonic and climatic signals from source to sedimentary archive. Nature Geoscience, 4, 231-235, doi: 10.1038/ngeo1087}
\pub {4}{ {\bf Armitage, J.J.} \& Allen, P.A. (2010) Cratonic basins and the long-term subsidence history of continental interiors. Journal of the Geological Society, London, 167, 61-70, doi: 10.1144/0016-76492009-108}
\pub {3}{ {\bf Armitage, J.J.}, Collier, J.S., Minshull, T.A., (2010) The importance of rift history for volcanic margin formation. Nature, 465, 913-917, doi: 10.1038/nature09063}
\pub {2}{ {\bf Armitage, J.J.}, Henstock, T.J., Minshull, T.A., Hopper, J.R.,  (2009) Lithospheric controls on melt production during continental breakup at slow rates of extension: Application to the North Atlantic. Geochemistry Geophysics Geosystems, 10(Q06018), doi: 10.1029/2009GC002404}
\pub {1}{ {\bf Armitage, J.J.}, Henstock, T.J., Minshull, T.A., Hopper, J.R., (2008) Modelling the composition of melts formed during continental break-up of the North Atlantic. Earth Planetary Science Letters, 269, 248-258, doi: 10.1016/j.epsl.2008.02.024}

