\chapter{Example Publications Part 2: Surface}

In this final annex I have selected five example publications that explore how surface processes are recorded within the geological record. They also demonstrate the various numerical models I have developed and the collaborations I have formed in order to try and broaden my understanding of the many parts to sedimentology and geomorphology.

\begin{itemize}

\item [1] {\bf Transformation of tectonic and climatic signals from source to sedimentary archive -- Nature Geoscience}

This is where my interest in sedimentology all began. While at Imperial College I got distracted by the work my friends were doing on grain size deposition in alluvial fans. Encouraged by them I developed a novel model that could predict grain sizes deposited as a function of past tectonics and climate (precipitation). This model was used to then demonstrate that change in climate and tectonics leave diagnostically different signatures in the landscape.

\item[2] {\bf Physical stratigraphic model for the Eocene Escanilla sediment routing system: Implications for the uniqueness of sequence stratigraphic architectures -- Journal of Sedimentary Research}

Perhaps my longest title. With funding from the Royal Astronomical Society, I tried to see if I could use the model published in Nature Geosceinces in 2011 to invert the sedimentary record for past climate change. This publication was the result of that work, where the position of key sedimentological markers, such as the gravel front, were used to invert for change in run-off and hence pricipitation.

\item[3] {\bf Deciphering the origin of cyclical gravel front and shoreline progradation and retrogradation in the stratigraphic record -- Basin Research}

Once I had found that my methodology worked for terrestrial deposits I decided to push beyond the shoreline, and explore how the gravel front and shoreline are impacted by climate change and sea-level change. In this study I demonstrate that both a change in run-off and sea level can give similar stratigraphic responses. The difference between the two is recorded in the most proximal regions, suggesting that to fully understand stratigraphy we need complete sections from terrestrial coarse gravel deposits down to marine mud and sand.

\item[4] {\bf  Numerical modelling landscape and sediment flux response to precipitation rate change -- Earth Surface Dynamics}

Landscape evolution models (LEMs) for geological processes tend to treat erosion as either a kinematic wave equation or a diffusive process. In this study I explored the implications of htese two end-member models on the response to change in surface run-off. I found that both models gave similar response, suggesting that it will be difficult to decipher which assumptions are relavent for modelling past landscape change. However, the models both suggest landscape responds more rapidly to a wetting event than a drying event, suggesting that short lived gravel deposition is a signature of increased surface run-off.

\item[5] {\bf Short communication: flow as distributed lines within the landscape -- Earth Surface Dynamics}

While developing the above mentioned study, I noticed that some LEMs where highly resolution dependent. That is to say that the numerical result was a function of the grid resolution. In this final paper I explored the causes of this resolution dependence, and found that how surface water is routed down slope plays a key role in numerical model stability. By routing flow down all slopes I developed a LEM that is not resolution dependent. It remains to be seen if the community will notice.

\end{itemize}
