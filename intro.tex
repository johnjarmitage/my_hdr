\chapter{Introduction}

This thesis for the `habilitation à diriger des recherches' is split into four parts: a summary of my past research, a summary of my future research outlook, my résumé, and ten examples of my research publications. In this document I wish to explore how geological observations reflect processes that occur deep within the planet or deep with in time. In Earth science we are faced with a great difficulty, there are rarely experiments that can recreate the processes that we which to understand. This means we cannot test hypothesis with experiments, but instead we must attempt to infer how the Earth operates. Inference is not a science, rather it is reasoned argument, and this can lead to some interesting debates but also to a lack in scientific rigour.

\textcolor{blue}{Need to expand on this statement here. I cannot leave the last phrase hanging.}

My working life has been spent, for the majority of the time, in trying to develop methods that allow hypothesis to be tested against observations. Throughout my career I have tried to replace the impossible experiment with numerical models hat approximate the fundamental processes and predict geological observations. These tools allow inferences, interpretations, to be tested. I have worked with observationas from seismology to sedimentology and the central core of my work has been to test how the Earth works against the information left behind in the rocks.

Why are silicic sediments preserved within Antarctic deep sea sediments? This is the question that started my research career. Deep sea sediments around Antarctica are anomalously high in silica content, but is this due to enhanced preservation or enhanced productivity? For a few months I explored the dissolution rates of diatamoceous sediments collected from the Antarctic shelf sea environment, and explored the structure of the diatoms using a scanning electric microscope and Fourier transform infra-red microscopy. No scientific publications came of this study. However, I learnt two things: (1) I am not good in the lab, and (2) I wanted to stay in research. Both remain true today.

My past research is broadly split into two areas. I have focused some time on trying to understand how mantle melting is expressed in surface observations. I have also spent an equal amount of time trying to work out how past climatic change is recorded within sedimentary deposits. It is for this reason that I gave the title `Rivers of Melt'. I will roughly divide this thesis into two parts, mantle and surface. I will subsequently pull these two parts together and present the direction I hope my work will take in the coming years.
