%%%%%%%%%%%%%%%%%%%%%%%%%%%%%%%%%%%%%%%%%
% Wilson Resume/CV
% XeLaTeX Template
% Version 1.0 (22/1/2015)
%
% This template has been downloaded from:
% http://www.LaTeXTemplates.com
%
% Original author:
% Howard Wilson (https://github.com/watsonbox/cv_template_2004) with
% extensive modifications by Vel (vel@latextemplates.com)
%
% License:
% CC BY-NC-SA 3.0 (http://creativecommons.org/licenses/by-nc-sa/3.0/)
%
%%%%%%%%%%%%%%%%%%%%%%%%%%%%%%%%%%%%%%%%%

%----------------------------------------------------------------------------------------
%	PACKAGES AND OTHER DOCUMENT CONFIGURATIONS
%----------------------------------------------------------------------------------------

\documentclass[hidelinks,10pt]{article} % Default font size
\usepackage{textcomp}
\usepackage{eurosym}

\input{structure.tex} % Include the file specifying document layout

%----------------------------------------------------------------------------------------

\begin{document}

%----------------------------------------------------------------------------------------
%	NAME AND CONTACT INFORMATION
%----------------------------------------------------------------------------------------

\title{John J.~Armitage -- Résumé} % Print the main header

%------------------------------------------------

\parbox{0.5\textwidth}{ % First block
\begin{tabbing} % Enables tabbing
\hspace{3cm} \= \hspace{4cm} \= \kill % Spacing within the block
{\bf Adresse} \> Institut de Physique du Globe \\ % Address line 1
\> de Paris, 1 rue Jussieu, Paris, 75005 \\ % Address line 2
{\bf Date de Naissance} \> 17$^{th}$ Juin 1981 \\ % Date of birth 
{\bf Nationalité} \> Britannique % Nationality
\end{tabbing}}
\hfill % Horizontal space between the two blocks
\parbox{0.5\textwidth}{ % Second block
\begin{tabbing} % Enables tabbing
\hspace{3cm} \= \hspace{4cm} \= \kill % Spacing within the block
{\bf Téléphone} \> +33 (0)1 83 95 78 12  \\ % Home phone
%{\bf Mobile Phone} \> +33 (0)6 80 32 76 52 \\ % Mobile phone
{\bf Émail} \> \href{mailto:armitage@ipgp.fr}{armitage@ipgp.fr} \\ % Email address
{\bf Langues} \> Anglais, Français (niveau B1) \\
\end{tabbing}}

%----------------------------------------------------------------------------------------
%	PERSONAL PROFILE
%----------------------------------------------------------------------------------------
%
%\section{Personal Profile}
%
Je suis intéressé à identifier les principaux contrôles physiques sur les processus géologiques. J'ai concentré mes recherches sur les relations entre différents observables: par exemple, le taux de production de magma par rapport au taux de rifting, la structure de la vitesse sismique par rapport à la distribution du magma, et le transport des sédiments par rapport à la taille des grains. Le cœur de mon travail consiste à comprendre la relation entre des processus physique et des observations au sein de la science de la Terre.
%
%----------------------------------------------------------------------------------------
%	EDUCATION SECTION
%----------------------------------------------------------------------------------------

\section{Éducation}

\tabbedblock{
\bf{2004-2009} \> Doctorat en géophysique - National Oceanography Centre, Southampton, Royaume Uni\\[5pt]
\>Titre: How Does Melting Affect Continental Rift Development? \\
\>Superviseurs: Prof Tim Minshull, Dr Tim Henstock \& Dr John Hopper\\
\>\+
\textit{Avec une bourse d'études du Natural Enviroment Research Council}
}

%------------------------------------------------

\tabbedblock{
\bf{2003-2004} \> Masters en océanographie - National Oceanography Centre, Southampton, Royaume Uni\\[5pt]
\>Titre du projet de masters: Dissolution of Silica from Antarctic Continental Shelf Sea Sediments\\ \>Superviseur: Prof. Rachel Mills\\
\>\+
\textit{Avec une bourse d'études du Natural Enviroment Research Council}
}

%------------------------------------------------

\tabbedblock{
\bf{1999-2003} \> Msci (Licence + 1) en Physique - Imperial College London, Royaume Uni \\%[5pt]
%\>Upper Second Class Honours- 79\% Average\\
\>\+
}

%----------------------------------------------------------------------------------------
%	EMPLOYMENT HISTORY SECTION
%----------------------------------------------------------------------------------------

\section{Employment History}

\job
{Sep 2015 -}{Présent}
{Institut de Physique du Globe de Paris, France}
{http://www.ipgp.fr}
{ANR Accueil de Chercheurs de Haut Niveau}
{Je suis le chercheur principal de Interift, un projet de recherche de 4 ans financée par l'ANR à travers l'appel «Accueil de Chercheurs de Haut Niveau». Le projet vise à comprendre comment se forment les bassins continentaux. Actuellement, il existe un décalage entre les observations géochimiques et sismologiques de la structure de la lithosphère et de l'asthénosphère au-dessous du Rift Est-Africain. L'objectif principal de ce projet est de développer des modèles géophysiques de la déformation du manteau et de prédire les effets de transport du magma sur la structure sismique et géochimique de la Terre.}

%------------------------------------------------

\job
{Sep 2013 -}{Aug 2015}
{Department of Earth Sciences, Royal Holloway, University of London, Royaume Uni}
{https://www.royalholloway.ac.uk/earthsciences/home.aspx}
{Royal Astronomical Society Research Fellow}
{Ce projet de recherche, intitulé «Deciphering the sedimentary record: tectonic versus climate change», a été décernée par la Royal Astronomical Society. J'ai exploré la façon dont la topographie varie en raison de l'écoulement du manteau, et si les changements climatiques de longue durée ont affecté les archives sédimentaires à l'intérieur des continents. Je continue des collaborations avec Prof.~Jason Phipps Morgan (Royal Holloway), Prof.~Peter Burgess (University of Liverpool) et Dr.~Marta Pérez-Gussinyé (MARAUM, Brême) pour coupler des modèles numériques de l'écoulement du manteau aux processus sédimentaires.}

%------------------------------------------------

\job
{Apr 2011 -}{Aug 2013}
{Institut de Physique du Globe de Paris, France}
{http://www.ipgp.fr}
{Marie Curie Research Fellow}
{Ce projet, financé à travers l'appel à projet «Marie Curie» de l'UE, s'intitulait «Cratonic basins: an archive of lithosphere-mantle interaction». En collaboration avec Prof.~Claude Jaupart (IPGP), j'ai étudié la façon dont la dynamique du manteau est enregistrée dans les roches sédimentaire. J'ai étudié la relation entre la déstabilisation de la lithosphère et l'affaissement dans l'intérieur continental.}

%------------------------------------------------

\job
{Mar 2008 -}{Mar 2011}
{Department of Earth Science and Engineering, Imperial College London, Royaume Uni}
{https://www.imperial.ac.uk/earth-science/}
{Chercheur Contractuel}
{J'ai travaillé au sein de l'équipe des systèmes sédimentaires, diriger par Prof.~Philip Allen. J'ai développé des modèles d'érosion des bassins versants et de dépôts de sédiments pour explorer les contrôles de la stratigraphie des bassins dus aux changements de climat et de soulèvement dans le bassin versant.}

%----------------------------------------------------------------------------------------
%	FUNDING
%----------------------------------------------------------------------------------------

\section{Financements}

\tabbedblock{
\bf{Sep 2015} \> Appel «Accueil de Chercheurs de Haute Niveau», ANR; projet: \\
\>  «InterRift: Interpreting Continental Break-up from Surface Observations» -- 450,000\euro \\
\> (Investigateur Principale) \\
\bf{Aug 2013} \> Royal Astronomical Society Research Fellowship -- \pounds123,000 \\
\> (Investigateur Principale) \\
\bf{Apr 2011} \> E.U.~Marie Curie Research Fellowship -- 193,564\euro \\
\> (Investigateur Principale) \\
}

%----------------------------------------------------------------------------------------
%	AWARDS
%----------------------------------------------------------------------------------------

\section{Prix}

\tabbedblock{
\bf{2011} \> Young Author of the Year, Journal of the Geological Society, London, pour le article\\
\> «Cratonic basins and the long-term subsidence history of continental interiors».\\
\bf{2009} \> Masanori Sakuyama Prize, pour le meilleur ouvrage d'une thésard d'National \\
\> Oceanography Centre, Southampton: «Modelling the composition of melts formed during the\\
\> continental break-up of the North Atlantic», publié dans \textit{Earth and Planetary Science Letters}.\\
\bf{2007} \> Meilleur poster presenté au Science \& Mathematics Research\\
\> Showcase, Faculty of Engineering, University of Southampton.\\
\bf{2003} \> NERC scholarship pour le Masters en Océanographie et pour mon doctorant.\\
}

%----------------------------------------------------------------------------------------
%	PhD SUPERVISION
%----------------------------------------------------------------------------------------

\section{Supervision des doctorantes}

\job
{2016 -}{2019}
{Thijs Franken}
{}
{Institut de Physique du Globe de Paris}
{Je suis le co-directeur de la thèse de Thijs Franken. Le doctorat de Thijs se concentrera sur l'élaboration de modèles de la structure sismique en raison du transport du magma et de la propagation des ondes par des domaines régionaux. Ce projet est financé par ma subvention ANR et est en collaboration avec Nobuaki Fuji et Alexandre Fournier (IPGP).}

\job
{2014 -}{2018}
{Sam Brooke}
{}
{Imperial College London}
{Je suis le co-directeur de la thèse de Sam Brooke avec Alex Whittaker au Imperial College London. Le but de ce projet est de développer davantage les modèles de transport de sédiments et l'évolution des cônes alluviaux.}

\job
{2012 -}{2017}
{Chandra Taposeea}
{}
{Imperial College London}
{J'ai etait le co-directeur de la thèse de Chandra Taposeea avec Jenny Collier à l'Imperial College London. Elle travaille maintenant avec isardSAT, une entreprise qui propose des services dans le champ de l'observation de la Terre.}

%----------------------------------------------------------------------------------------
%	MASTERS SUPERVISION
%----------------------------------------------------------------------------------------
\section{Supervision de projets du masters et licence}

\tabbedblock{
\bf{2017} \> M2 stage: Aimen Maghrebi, \textit{Ecole Supérieure d’Ingénieurs Léonard de Vinci}.\\
\bf{2017} \> L3 stage: Sabrina Ihaddadene, Licence, \textit{Physique, Université Paris-Est Créteil}.\\
\bf{2016} \> L3 stage: Leo Bourier, Licence, \textit{Physique, Université Paris-Est Créteil}.\\
}

%----------------------------------------------------------------------------------------
%	SELECTED CONFERENCE PRESENTATIONS
%----------------------------------------------------------------------------------------

\section{Présentations Invités}

{\bf EGU General Assembly 2018}: From induction to deduction: Using the Earth as a natural laboratory. EGU2018-10469, avril, 2018.\\
{\bf Ordinary Meeting of the Royal Astronomical Society}: Can variations in the Earth's orbit create stratigraphic sequences? 10th mars 2017\\
{\bf Geological Society of London: Rifts III:} Catching the wave, 2016: Upper mantle temperature during extension and breakup, 22-24 mars, 2016.\\
{\bf AGU Fall Meeting 2015}: Testing How Depletion, Dehydration and Melt Affect Seismic Expressions of the Asthenosphere, T34C-01, 16 decembre, 2015.\\
{\bf Volkswagen-Stiftung Funded Symposium}: Bridging the Gap Between Field Evidence and Numerical Models: Modelling landscape and sediment flux responses to precipitation change, Herrenhausen Palace Conference Centre, Hanover, 21-23 octobre, 2015.\\
{\bf Action Marges Workshop}: Mouvements verticaux et Chantier Afar-Aden: Keynote: Controls of lithospheric extension on magma and sediment production, Total, La Defence, 11-12 novembre, 2014.\\
{\bf Volcanic \& Magmatic Studies Group 2012 Conference}: Keynote: Beyond Hotspots: the importance of rift history for volcanic margin formation, Durham University, UK, 4-6 janvier, 2012.\\
{\bf EarthScope - GeoPRISMS Science Workshop on Eastern North America (ENAM)}: Analogue and numerical models that inform the rifting process, Lehigh University, Bethlehem, PA, USA, 27-29 octobre, 2011.\\

\section{Seminaires Invités}
Plus de 20 séminaires invités depuis 2012.
\tabbedblock{
\bf{2018} \> University of Leeds \\
\bf{2017} \> ISTerre Grenoble; GET Toulouse; IRAP Toulouse \\
\bf{2016} \> Universié de Lausanne; GFZ Potsdam \\ 
\bf{2015} \> University of Edinburgh; MINES ParisTech; University of Southampton \\
\bf{2014} \> University of Southampton \\
\bf{2013} \> Université de Montpelier; Université de Nantes; Université Paris Sud, Orsay \\
\bf{2012} \> ENS, Paris; FAST, Université Paris Sud, Orsay; CPRG Nancy; UMPC, Paris; Université de Rennes;\\
          \> EOST Strasbourg \\
}

%\section{Selected Oral Conference Presentations}
%Armitage, J.J. (2017), The Effect of Climate Change on Volcanism During Continental Break-up, 79th EAGE Conference and Exhibition 2017.\\
%Armitage, J.J. (2016), But What About Trees and Beavers? Simplicity, Complexity and Benchmarks for Landscape Evolution Models,  AGU, Fall Meet., Abstract, EP12A-07\\
%Armitage, J.J. (2015), Landscape response due to sediment transport and bedrock detachment, GeoBerlin, Dynamic earth – from Alfred Wegener to today and beyond, Berlin.\\
%Armitage, J.J. (2014), The influence of large-scale tectonics, mantle flow and climatic change on sediment accumulation, William Smith Meeting 2014: The Future of Sequence Stratigraphy: Evolution or Revolution? The Geological Society, London.\\
%Armitage J.J., Barrier, L. (2013), Is the Neogene series of the Northern Foreland Basin of the Tian Shan Range indicative of tectonic or climatic change? AGU, Fall Meet., Abstract EP43E-04 \\
%Armitage J.J. (2012), The Instability of Continental Passive Margins and its Effect on Continental Topography and Heat Flow, Deep Water Margins, The Geological Society, London.\\
%Goes, S.D., Armitage, J.J., Harmon, N., Huismans, R.S., (2011), Low seismic velocities below mid-ocean ridges: Attenuation vs. melt retention, AGU, Fall Meet., Abstract, T33H-05 \\
%Armitage J.J., Duller, R.A., Whittaker, A.C., Densmore, A, Allen, P.A., (2010) Response of sediment routing systems to tectonic and climatic perturbations, William Smith Meeting 2010 - Landscapes Into Rock, The Geological Society, London.\\
%Armitage J.J., Collier, J.S., Minshull T.A., Hopper J.R., (2008), Geodynamic modelling of the opening of the northwest Indian Ocean, AGU, Fall Meet., Abstract, T53G-08\\
%Armitage J.J., Henstock T.J., Minshull T.A., Hopper J.R., (2007), Lithospheric controls on the rifting of continents at slow rates of extension, AGU, Fall Meet. Suppl., Abstract, T32A-06\\

\clearpage
\pagebreak

%----------------------------------------------------------------------------------------
%	PUBLICATIONS
%----------------------------------------------------------------------------------------

\section{Liste des publications}

\pub {28}{ {\bf Armitage, J.J.}, Petersen, K.D., Perez-Gussinye, M., (2018) The Role of Crustal Strength in Controlling Magmatism and Melt Chemistry During Rifting and Break-Up. Geochemistry Geophysics Geosystems, doi: 10.1002/2017GC007326}
\pub {27}{ {\bf Armitage, J.J.}, Whittaker, A.C., Zakari, M., Campforts, B., (2018) Numerical modelling landscape and sediment flux response to precipitation rate change. Earth Surface Dynamics, 6, 77-99, doi: 10.5194/esurf-6-77-2018}
\pub {26}{ {\bf Armitage, J.J.}, Burgess, P.M.,  Hampson, G.J., Allen, P.A., (2018) Deciphering the origin of cyclical gravel front and shoreline progradation and retrogradation in the stratigraphic record. Basin Research, 30, 15-30, doi: 10.1111/bre.12203}
\pub {25}{ {\bf Armitage, J.J.}, Collier, J.S., (2017) The thermal structure of volcanic passive margins. Petroleum Geoscience, doi: 10.1144/petgeo2016-101}
\pub {24}{ Temme, A.J.A.M., {\bf Armitage, J.J.}, Attal, M., van Gorp, W., Coulthard, T.J., Schoorl, J.M., (2017) Developing, choosing and using landscape evolution models to inform field-based landscape reconstruction studies. Earth Surface Processes and Landforms, 42, 2167–2183, doi: 10.1002/esp.416}
\pub {23}{ Mareschal, J.-C., Jaupart, C., {\bf Armitage, J.J.}, Phaneuf, C., Pickler, C.,  Bouquerel, H., (2017) The Sudbury Huronian Heat Flow Anomaly, Ontario, Canada. Precambrian Research, 295, 187-202, doi: 0.1016/j.precamres.2017.04.024}
\pub {22}{ Taposeea, C.A., {\bf Armitage, J.J.}, Collier, J.S., (2017) Asthenosphere and lithosphere structure controls on early onset oceanic crust production in the southern South Atlantic. Tectonophysics, 716, 4-20, doi 10.1016/j.tecto.2016.06.026}
\pub {21}{ Allen, P.A., Michael, N.A., D’Arcy M., Roda-Boluda, D.C., Whittaker, A.C., Duller, R.A, {\bf Armitage, J.J.}, (2017) Fractionation of grain size in terrestrial sediment routing systems. Basin Research, 29, 180-202, doi: 10.1111/bre.12172}
\pub {20}{ Duller, R.A., Warner, N.H., De Angelis, S., {\bf Armitage, J.J.}, Poyatos-More, M., (2015) Reconstructing the timescale of a catastrophic fan-forming event on Earth using a Mars model. Geophysical Research Letters, 42, 10324-10332, doi: 10.1002/2015GL066031}
\pub {19}{ {\bf Armitage, J.J.}., Allen, P. A., Burgess, P. M., Hampson, G. J., Whittaker, A. C., Duller, R. A., Michael, N. A., (2015) Physical stratigraphic model for the Eocene Escanilla sediment routing system: Implications for the uniqueness of sequence stratigraphic architectures. Journal of Sedimentary Research, 85, 1510-1524, doi:10.2110/jsr.2015.97}
\pub {18}{ Allen, P.A., {\bf Armitage, J.J.}, Whittaker, A.C., Michael, N.A., Roda-Boluda, D., D’Arcy, M., (2015) Fragmentation model of the grain size mix of sediment supplied to basins. Journal of Geology, 123, 405-427, doi: 10.1086/683113}
\pub {17}{ {\bf Armitage, J.J.}, Ferguson D., Goes, S., Hammond, J.O.S., Calais, E., Harmon, N., Rychert, C.A., (2015) Upper mantle temperature and the onset of extension and break-up in Afar, Africa. Earth and Planetary Science Letters, 418, 78-90, doi: 10.1016/j.epsl.2015.02.039}
\pub {16}{ Lucazeau, F., {\bf Armitage, J.J.}, Kadima Kabongo, E., (2015) Thermal regime and evolution of the intracratonic Congo Basin, in The Geology and Resource Potential of the Congo Basin , de Wit, M.J., Guillocheau, F., Fernadez-Alonso, M., Kanda, N., and de Wit, M.C.J., Eds. Elsevier. doi: 10.1007/978-3-642-29482-2\_12}
\pub {15}{ Petersen, K.D., {\bf Armitage, J.J.}, Nielsen, S.B., Thybo, H., (2015) Mantle temperature as primary control on the time scale of thermal evolution of extensional basins, Earth and Planetary Science Letters, 409, 61-70, doi: 10.1016/j.epsl.2014.10.043}
\pub {14}{ {\bf Armitage, J.J.}, Duller, R.A., Schmalholz, S.M., (2014) The influence of long-wavelength tilting and climatic change on sediment accumulation. Lithosphere, 6, 303-318, doi: 10.1130/L343.1}
\pub {13}{ {\bf Armitage, J.J.}, Dunkley Jones, T., Duller, R.A., Whittaker, A.C., Allen, P.A., (2013) Temporal buffering of climate-driven sediment flux cycles by transient catchment response. Earth and Planetary Science Letters, 369, 200-210, doi: 10.1016/j.epsl.2013.03.020}
\pub {12}{ {\bf Armitage, J.J.}, Jaupart, C., Fourel, L., Allen, P.A. (2013) The instability of continental passive margins and its effect on continental topography and heat flow. Journal of Geophysical Research – Solid Earth, 118, 1817–1836, doi: 10.1002/jgrb.50097}
\pub {11}{ Allen, P.A., {\bf Armitage, J.J.}, Carter, A., Duller, R.A., Michael, N., Sinclair, H.D., Whitchurch, A.L., Whittaker, A.C. (2013) The Qs problem: Sediment mass balance of proximal foreland basin systems. Sedimentology, 60, 102-130, doi: 10.1111/sed.12015}
\pub {10}{ Goes, S., {\bf Armitage, J.J.}, Harmon, N., Smith, H., Huismans, R., (2012) Low seismic velocities below mid-ocean ridges: Attenuation vs. melt retention. Journal of Geophysical Research – Solid Earth, 117(B12403), doi: 10.1029/2012JB009637}
\pub {9}{ Duller, R.A., Whittaker, A.C., Swinehart, J.B., {\bf Armitage, J.J.}, Sinclair, H.D., Bair, A.R., Allen, P.A., (2012) Abrupt landscape change post-6 Ma on the Central Great Plains, U.S.A. Geology, 40, 871-874}
\pub {8}{ Allen, P.A. \& {\bf Armitage, J.J.}, (2012) Cratonic basins. In Busby, C. \& Azor, A. (Eds.) Syntectonic Basin Development, Active to Ancient: Recent Advances. Ch. 30, p 602-620, Blackwell Publishing Ltd.}
\pub {7}{ {\bf Armitage, J.J.}, Warner, N.H., Goddard, K., Gupta, S., (2011) Timescales of alluvial fan development on Mars. Geophysical Research Letters , 38 (L17203), doi:10.1029/2011GL048907}
\pub {6}{ {\bf Armitage, J.J.}, Collier, J.S., Minshull, T.A. Henstock, T.J. (2011) Thin oceanic crust and flood basalts: Reconciling observations from the northwest Indian Ocean. Geochemistry Geophysics Geosystems, 12(Q0AB07), doi:10.1029/2010GC003316}
\pub {5}{ {\bf Armitage, J.J.}, Duller, R.A., Whittaker, A.C. Allen, P.A., (2011) Transformation of tectonic and climatic signals from source to sedimentary archive. Nature Geoscience, 4, 231-235, doi: 10.1038/ngeo1087}
\pub {4}{ {\bf Armitage, J.J.} \& Allen, P.A. (2010) Cratonic basins and the long-term subsidence history of continental interiors. Journal of the Geological Society, London, 167, 61-70, doi: 10.1144/0016-76492009-108}
\pub {3}{ {\bf Armitage, J.J.}, Collier, J.S., Minshull, T.A., (2010) The importance of rift history for volcanic margin formation. Nature, 465, 913-917, doi: 10.1038/nature09063}
\pub {2}{ {\bf Armitage, J.J.}, Henstock, T.J., Minshull, T.A., Hopper, J.R.,  (2009) Lithospheric controls on melt production during continental breakup at slow rates of extension: Application to the North Atlantic. Geochemistry Geophysics Geosystems, 10(Q06018), doi: 10.1029/2009GC002404}
\pub {1}{ {\bf Armitage, J.J.}, Henstock, T.J., Minshull, T.A., Hopper, J.R., (2008) Modelling the composition of melts formed during continental break-up of the North Atlantic. Earth Planetary Science Letters, 269, 248-258, doi: 10.1016/j.epsl.2008.02.024}


\end{document}
