%%%%%%%%%%%%%%%%%%%%%%%%%%%%%%%%%%%%%%%%%%%%%%%%%%%%%%%%%%%%%%%%%%%%%%
%Package These (dans ~/Styles pour non-standards)
%%%%%%%%%%%%%%%%%%%%%%%%%%%%%%%%%%%%%%%%%%%%%%%%%%%%%%%%%%%%%%%%%%%%%%


%\usepackage[pagebackref]{hyperref}


\usepackage[a4paper,twoside,textwidth=14cm]{geometry}
\usepackage{amsmath,amsfonts,amssymb}      % classique
%\usepackage{frbib}                        % bib en fran{\c c}ais : fralpha
\usepackage{picins}                        % fig ds paragraphe
\usepackage{bm}                            % pour lettre bold en math
\usepackage{latexsym}                      % Latex symbole (pour \Box)
\usepackage{stmaryrd}                      % pour llbracket ...
\usepackage{url}                           % pour URL
\usepackage[francais]{babel}               
\usepackage[french]{minitoc}               % mini table des mati{\`e}res
\usepackage{graphics}                      % pour les graphics
\usepackage{epsfig}
\usepackage{fancyhdr}                      % en tete, pied de page
\usepackage{subfigure}                     % sous figures
\usepackage{colortbl}                      % couleur ds les tableaux
\usepackage{floatflt}                      % figure ds le texte
\usepackage{placeins}                      % placement flottants (\FloatBarrier)

\usepackage[french,vlined,ruled]{algorithm2e}


%\usepackage{cmbright}                   % sty pour la font


\usepackage{authorindex}                   % index des auteurs
%\let\cite=\aicite  
                        % pour le package pr{\'e}c{\'e}dent
%\usepackage{natbib}

\usepackage{multicol}

\usepackage[grey,utopia]{quotchap}         % pour num{\'e}ros chapitres en gris
\usepackage[twoside,figuresleft]{rotating} % turn table (pas dvi que ps)
%\usepackage{french}                        % langue : javanais
\usepackage[T1]{fontenc}                   % pour coupure d'accent
\usepackage{index}
%%Pour KDVI
\usepackage[active]{srcltx}
\usepackage{colortbl}
\usepackage{color}
\usepackage[bookmarks,pagebackref]{hyperref}

%%%% Commun options
\hypersetup{
  linktocpage,%
  %%------------- Color Links ------------------------------ 
  colorlinks=true,% 
  linkcolor=myred,%
  citecolor=mydarkblue,% 
  urlcolor=myblue,%
  menucolor=red,%
  %%------------- Doc Info --------------------------------- 
  pdftitle={HDR},%
  pdfauthor={David Coeurjolly},%
  %%------------ Doc View ----------------------------------}
  pdfhighlight=/P,%
  bookmarksopen=false,%
  plainpages=false,
  pdfpagelabels,
  pdfpagemode=None}

\hyperbaseurl{http://liris.cnrs.fr/david.coeurjolly/}

\definecolor{myblue}{rgb}{0,0,0.7}
\definecolor{myred}{rgb}{0.7,0,0}
\definecolor{mygreen}{rgb}{0,0.7,0}
\definecolor{mydarkblue}{rgb}{0,0,0.5} 


%%%%%%%%%%%%%%%%%%%%%%%%%%%%%%%%%%%%%%%%%%%%%%%%%%%%%%%%%%%%%%%%%%%%%%

\graphicspath{{./Figures/},{./FiguresNF/}}                    % chemin des figs
%listfiles                                 % liste des fichiers a la compil
%%%%%%%%%%%%%%%%%%%%%%%%%%%%%%%%%%%%%%%%%%%%%%%%%%%%%%%%%%%%%%%%%%%%%%
\pagestyle{fancy}                          % pour fancychap
\fancyhf{}                                 % on vide les pieds de pages
\fancyhead{}                               % on vide les en-tete
\fancyhead[RO]{\slshape \rightmark}        % droite, page paire : section
\fancyhead[LE]{\slshape \leftmark}         % gauche, page impaire : chapitre
\fancyfoot[C]{-~\thepage~-}                % centre bas, num{\'e}ro de page

\renewcommand{\chaptermark}[1]{%           % chapitre en minuscule
  \markboth{\chaptername~%
    \thechapter.\ #1}{}}
\renewcommand{\sectionmark}[1]{%           % section en minuscule
  \markright{\thesection.\ #1}{}}


\fancypagestyle{plain}{%                   % red{\'e}finition style `plain'
  \fancyhf{}\fancyhead{}                   % rien en haut ni en bas
  \cfoot{-~\thepage~-}                     % centre bas, num{\'e}ro de page
  \renewcommand{\headrulewidth}{0pt}       % pas de ligne en haut
  \renewcommand{\footrulewidth}{0pt}       % pas de ligne en bas
 }


%%%%%%%%%%%%%%%%%%%%%%%%%%%%%%%%%%%%%%%%%%%%%%%%%%%%%%%%%%%%%%%%%%%%%%
\newenvironment{mapreuve}%                 % environnement preuve
{\begin{description}%               
\item [Preuve :] \sl }{ \hfill $\Box$%     % termine par carr{\'e} blanc
\end{description} }

\newtheorem{theo}{Th{\'e}or{\`e}me}[chapter]           % environnement th{\'e}or{\`e}me
\newlength{\xvtextwidth}
\xvtextwidth\textwidth 
\advance\xvtextwidth - 4cm

%\renewcommand{\emph}[1]{#1\scalebox{0.4}{\includegraphics{dav.eps}}}

\newtheorem{defi}{D{\'e}finition}[chapter]
\newtheorem{lem}{Lemme}[chapter]
\newtheorem{coro}{Corollaire}[chapter]
\newtheorem{prop}{Proposition}[chapter]
\newtheorem{conj}{Conjecture}[chapter]
\newtheorem{example}{Exemple}[chapter]
%%%%%%%%%%%%%%%%%%%%%%%%%%%%%%%%%%%%%%%%%%%%%%%%%%%%%%%%%%%%%%%%%%%%%%
 % \nomtcrule
% \renewcommand{\mtctitle}{Sommaire\hrule}

\newcommand{\mychaptoc}[1]{%               % chapitre+minitoc
  \chapter{#1}                         
  \thispagestyle{plain}
  %\centerline{\Large $\Diamond$}
  \minitoc
  %\centerline{\Large $\Diamond$}
  %\flushright$\Box$
  \newpage}                                % saut de page

\newcommand{\mychaptocbis}[1]{%               % chapitre+minitoc
  \chapter*{#1}                         
  \thispagestyle{plain}
  %\centerline{\Large $\Diamond$}
  \minitoc
  %\centerline{\Large $\Diamond$}
  %\flushright$\Box$
  \newpage}                                % saut de page

%%%%%%%%%%%%%%%%%%%%%%%%%%%%%%%%%%%%%%%%%%%%%%%%%%%%%%%%%%%%%%%%%%%%%%
\setcounter{secnumdepth}{3}      % depth of numbering of sectionning commands
\setcounter{tocdepth}{1}         % depth of table of contents
\raggedbottom                    % or \flushbottom, at your choice
\setcounter{lofdepth}{1}         % pour la liste de figures : subfig aussi
\setcounter{minitocdepth}{3}     % profondeur minitoc



%%%%%%%%%%%%%%%%%%%%%%%%%%%%%%%%%%%%%%%%%%%%%%%%%%%%%%%%%%%%%%%%%%%%%%
\newcommand{\round}[1]{\lceil #1 \rfloor}  % notation arrondi
\def\eme{$^{\textrm{{\`e}me}}$}                  % i {\`e}me
\def\num{n^{\circ}}                        % numero
\def\Num{N^{\circ}}                        % Numero
\def\sinc{\mathrm{sinc}}                   % sinus cardinal
\def\ere{$^{\textrm{{\`e}re}}$}                % {\`e}re
\def\er{$^{\textrm{{e}r}}$}                % {\`e}re
\def\eg{\emph{e.g.} }                      % e.g.
\def\ie{\emph{i.e.} }                      % i.e.
\def\etc{\emph{etc}}                       % etc
\def\cm{\,cm}                              % cm
\def\met{\,m}                              % m
\def\mm{\,mm}                              % mm
\def\deg{$^\circ$}                         % degres
\def\ud{\mathrm{d}}                        % pour dx dy ...
\newcommand{\ball}  {\ensuremath{B}}

\newcommand{\MAset}{\ensuremath{\mathrm{A\!M}} }
\newcommand{\MAsetg}{\ensuremath{\MAset^g } }

\def \PS {{\aut{Planar-4-3-SAT}}}
\def \R {{\Bbb R}}
\def \I {{\Bbb I}}
\def \F {{\Bbb F}}
\def \S {{\Bbb S}}
\def \Z {{\mathbb Z}}
\def \L {{\mathcal{L}}}
\def \C {{\mathcal C}}
\def \P {{\mathcal P}}
\def \Q {{\mathcal Q}} 
\def \E{{\mathcal E}}
\def \D{{\mathcal D}}
\def \BD {{\bar{\mathcal{D}}}}
\def \etal {{\it et al.~}}
\def\arc{\mbox{arc}}
\newcommand{\B}   {\ensuremath{B}^{\text{<}}} %%  {\ensuremath{B}^{<}}
\newcommand{\AMDR}{\operatorname{AMD}}
\newcommand{\AMD}{\operatorname{AMD}}

\newcommand{\ffup}[1]{\uparrow#1\uparrow}
\newcommand{\fdown}[1]{\downarrow#1\downarrow}
\newcommand{\sI}[1]{\overline{\tt #1}}
\newcommand{\iI}[1]{\underline{\tt #1}}



\newcommand{\mat}[1]{\left( \begin{array}{cccc} #1 \end{array} \right)}
\newcommand{\matdd}[1]{\left( \begin{array}{cc} #1 \end{array} \right)}
\newcommand{\tabfdiv}[3]{\mbox{#1}  & #2 & =  #3} 

%%%%%%%%%%%%%%%%%%%%%%%%%%%%%%%%%%%%%%%%%%%%%%%%%%%%%%%%%%%%%%%%%%%%%%

\usepackage{pstchap}



%\newcommand{\monchapter}[1]{%%
%
%  \chapter{#1}%
%
% \vfill\minitoc\vfill\vfill

%\clearpage}

%\newcommand{\moncitet}[1]{
%%   \citet{#1}
%   \index{\citeauthor{#1}}
%}
\renewcommand{\listtablename}{Table des tableaux}

%ICI
%\makeindex


%\newindex{cite}{ctx}{cnd}{Index des auteurs cit{\'e}s}
%\renewcommand{\citeindextype}{cite}




% C'est pour les index avec NatBib
%\citeindextrue
%%%%%%%%%%%
%%%%%%%%%%%%%%%%%%% MULTIBIB
%\usepackage{multibib}

%%font auteus biblio

%\newfont{\auteursfont}{cmfib8}
\newcommand{\aut}[1]{{\sc #1}}             % auteur en small capsu
\newcommand{\OldName}[1]{{\sc #1}}             % auteur en small capsu
\newcommand{\Name}[1]{{\sc #1}}             % auteur en small capsu

%\newcites{ouv,journaux,conf,nat}{
%  Ouvrages,%
%  Articles dans des journaux internationaux,%
%  Articles dans des confrences internationales,%
%  Articles dans des confrences nationales}




%%%%%%%%%%%%%%%%%%%%%%%%%%%%%%%%%%%%%%%%%%%%%%%%

%SPECIFIQUE C1V
\definecolor{mongris}{gray}{0.8}           % definition couleur grise
\newcommand{\dd}{\footnotesize $\Diamond$}



%%%%%%%%%%%%%%%%%%%%%%%%%%%%%%%%%%%%%%%%%%%%%%%%%%%%%%
\newcommand{\pr}[1]{\textsl{#1}}
\newenvironment{cours}%
{\begin{center}\begin{tabular}{|p{1cm}p{5cm}p{3cm}ll|}
    \hline
    \rowcolor{mongris} \textsc{Année} & \textsc{Filiére} & \textsc{Matiére} &
  \textsc{Resp.} & \textsc{Vol.}\\\hline\hline}%
  {\hline\end{tabular}\end{center}}
\newcommand{\co}[5]{#1 & #2 & #3 & #4 & #5\\}

\newenvironment{these}%
{\begin{center}\begin{tabular}{|p{1cm}lp{5cm}l|}
    \hline  
    \rowcolor{mongris} \textsc{Année} & \textsc{Nom} & \textsc{Titre
      de la thèse} &
  \textsc{Encadrement.}\\\hline\hline}%
  {\hline\end{tabular}\end{center}}
\newcommand{\myth}[4]{#1 & #2 & #3 & #4 \\}

\newcommand{\e}[5]{#1 & #2 & #3 & #4 & #5 \\}
\newcommand{\eh}[5]{\rowcolor{mongris} \text{#1} & \text{#2} &  \text{#3} &  \text{#4} & \text{#5}\\} 


\newenvironment{cours2}%
{\begin{center}\begin{tabular}{|p{1cm}lp{6cm}ll|}
    \hline
    \rowcolor{mongris} \textsc{Année} & \textsc{Filiére} & \textsc{Matiére} &
  \textsc{Resp.} & \textsc{Vol.}\\\hline\hline}%
  {\hline\end{tabular}\end{center}}

\newcommand{\support}[1]{\emph{\dd~Support de cours rèalisè}~: #1}

%%%%%%%%%%%%%%%%%%%%%%%%%%%%%%%%%%%%%%%%%%%%%%%%%%%%%%

\newtheorem{propriete}{\textsc{\textmd{Propriètè}}}[chapter]
\newtheorem{proposition}{\textsc{\textmd{Proposition}}}[chapter]
\newtheorem{theoreme}{\textsc{\textmd{Theoréme}}}[chapter]
\newtheorem{lemme}{\textsc{\textmd{Lemme}}}[chapter]
\newtheorem{remarque}{\textsc{\textmd{Remarque}}}[chapter]
\newtheorem{definition}{\textsc{\textmd{Dèfinition}}}[chapter]


%%% Local Variables: 
%%% mode: latex
%%% TeX-master: "these.tex"
%%% End: 
