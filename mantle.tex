\chapter{Example Publications Part 1: Mantle}

In this appendix I have selected five key publications from my research carear that exemplify my apporach and collaborations for understanding how upper mantle processes are reflected in surface observations, and how these processes interact to control how the Earth has evolved.

\begin{itemize}

\item[1] {\bf The importance of rift history for volcanic margin formation -- Nature}

The first paper was published in Nature in 2010 a year after completion of my thesis and represents the culmination of four years work with Tim Minshull and Jenny Collier. In this article we present the hypothesis that lithosphere strucutre controls volcanism during continental break-up. This is expressed by the different evolution of crustal thickness at two classic volcanic margins, the North Atlantic Igneous Provinc and the Seychelles, Laxmi Ridge and Deccan Traps.

\item[2] {\bf Upper mantle temperature and the onset of extension and break-up in Afar, Africa -- Earth and Planetary Science Letters}

In this study I aimed to reconicle diverging interpretations on the structure of the mantle below Afar. Based on teh igneous geochemistry, researchers had reasoned that the upper mantle below Afar must be hot, very hot. However, seismic observations pointed to the mantle being not cold, or at least not hotter than average. I developed the methods to predict simultaneously the igneous geochemistry and seismic structure to demonstrate, that within observatial error, the mantle below Afar is hotter than average.

\item[3] {\bf Thermomechanical implications of sediment transport for the architecture and evolution of continental rifts and margins -- Tectonics}

This study comes from a collaboration I was involved in while working at Royal Holloway, University of London. I worked closely with Miguel Andrés-Mart\'inez to develop a coupled model of surface and shallow mantle processes. This model is the first to fully couple sedimentary processes with crustal deformation. In this study Miguel demonstrates that sediment deposition will alter the evolution of rifted basins. More importantly, geologic interpretation of sedimentary layers is over simplstic, with sediment sourced from one margin ending up on the other after break-up.

\item[4] {\bf The importance of Icenlandic ice sheet growth and retreat on mantle CO$_2$ flux -- Geophyscial Research Letters}

Noting that surface processes has a key role in break-up, in collaboration with various colleagues, I beccame interested in how ice-sheet loading might have impacted melt migration and retention. In this study I developed a simple 1D model of melt migration and composition. This allowed the classic hypotheses that Holocene deglaciation caused increased volcanism in Iceland. In this interdisciplineary collaboration, we demonstrated with a range of geological observations, that this hypothesis is most likely true.

\item[5] {Thijs' paper!} 

\end{itemize}
